\documentclass{article}
\usepackage{graphicx}
\usepackage{hyperref}

\title{Assignment 4}
\author{Abbas Khan , Mariia Rybalka , Linara Adilova}
\begin{document}
\maketitle 
    
\section*{Task 4.4 (theoretical): TLS Cipher Suites}
\subsection*{TLS\_ECDHE\_RSA\_WITH\_AES\_128\_GCM\_SHA256}
ECDHE - Cipher suites using authenticated ephemeral ECDH key agreement. the client and server will agree on encryption keys using Ephemeral Elliptic Curve Diffie-Hellman. 
RSA - Cipher suites using RSA key exchange, authentication or either respectively. the client will verify that the key is valid using the RSA algorithm to communications.
AES128 GCM - cipher suites using 128 bit AES. AES in Galois Counter Mode (GCM). the actual encryption of my web browsing session will be performed
SHA256 - Ciphersuites using SHA256. the SHA algorithm will be used for securely hashing parts of the TLS messages.

\begin{enumerate}
\item Authentication: Elliptic curve Diffie-Hellman (ECDH) is an anonymous key agreement protocol that allows two parties, each having an elliptic curve public-private key pair, to establish a shared secret over an insecure channel. This shared secret may be directly used as a key, or to derive another key which can then be used to encrypt subsequent communications using a symmetric key cipher. It is a variant of the Diffie-Hellman protocol using elliptic curve cryptography. Ephemeral keys are temporary and not necessarily authenticated, so if authentication is desired, authenticity assurances must be obtained by other means.
\item Encryption
\item Integrity
\end{enumerate}

TLS RSA RC4 128 MD5

\cite{label}

\begin{thebibliography}{9}

\bibitem{label}
  Not forget
  \emph{\url{https://en.wikipedia.org/wiki/Hash-based_message_authentication_code}}
  WeLiveSecurity by ESET
  
  https://en.wikipedia.org/wiki/Elliptic_curve_Diffie%E2%80%93Hellman
  
\end{thebibliography}

\end{document} 
