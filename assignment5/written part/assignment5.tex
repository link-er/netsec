\documentclass{article}
\usepackage{graphicx}
\usepackage{hyperref}

\title{Assignment 5}
\author{Abbas Khan , Mariia Rybalka , Linara Adilova}
\begin{document}
\maketitle 

\section*{Task 5.1 (theoretical): Block Cipher Based MACs}
\subsection*{Part (a)}
\begin{itemize}
\item ECB gives encrypted with a shared secret blocks of a plain text. So basically this works as MAC for showing that the sender is the one, who knows shared secret and it can prove integrity because receiver can just decrypt the message and compare with received plain text. But by definition MAC has to be a short tag, and here it will be at best as long as the message itself. Also it is easily breakable because it saves the structure of the plaintext.
\item CBC is usually used for generating MAC, by sending the last block. It will contain information about all the message and will be changed if the message is changed and also it will be encrypted by the shared secret, so it supports authenticity. The problem is with lots of known attacks, such as 
\item CTR is better than CBC in means of possibility to parallelise calculations, but like with ECB we need the whole encryption to support integrity, because here again all the blocks are different. It supports authenticity  because of using shared secret for encryption. 
\end{itemize}
\subsection*{Part (b)}
\begin{itemize}
\item In ECB mode padding can make it even easier to find out the shared secret - if the last block will be a constant padding attacker can guess the structure of the padding and the key.
\item In CBC mode padding adds ability to attack it also - because it is easy to get the new message, that will have the same MAC. One just have to know two pairs of plain text and MAC and then he can use it to construct new plain text with same MAC.
\item In CTR mode padding does not really change the situation, as we XOR last padded block with random counter and nonce.
\end{itemize}

\section*{Task 5.2 (theoretical): RADIUS}
Used material - \cite{radius}.
\\
From the traffic dump, taken during running 'authme' command, can be concluded that the hosts that take part in the authentication process as RADIUS server 10.0.0.10 (orange) and as RADIUS client 10.0.0.5 (white). I was able to found packets with protocol RADIUS, one is from 10.0.0.5 to 10.0.0.10 and one packet back. In the information about the packets I could find following:
\begin{itemize}
\item UDP packet, information concerned with RADIUS protocol: code of the packet is Access-Request, identifier of the packet and Authenticator code - random 128 bit string, used as salt for securing the password of the user, passed in the packet. Attribute Value Pairs in the packet are following: User-Name that has requested for authentication (plain text), User-Password in encrypted form, NAS-IP-Address (Network Access Server = NAS), NAS-Port, Message-Authenticator field.
\item Also a UDP packet, with code Access-Accept and the same identifier as in the request packet. Also the Authenticator code is contained in it.
\end{itemize}
If try to send not existing username or wrong password, the code of the answer packet will be Access-Reject.
\\
Before RADIUS interaction 10.0.0.1 (hellgate) establishes TLS connection with Authenticator (RADIUS client), in our case 10.0.0.5 (white). This also can be seen from the traffic dump. Every time when 'authme' is run this connection will be established.

\section*{Task 5.3 (practical): RADIUS}
Data needed for brute-force attack on the shared secret is one Access-Request packet. As we know the password sent and we can get request Authenticator code from the packet, we can use User-Password attribute for trying to find shared secret. User-Password attributed created as following (in case of password shorter than 16 octets): 
$$zero\_padded\_password \ XOR \ MD5(secret \ || \ authenticator)$$
We used just '0000000000000000' (16 zeros) as a password, in order to get needed length of the password - no need to pad it with zeros and split it to 16 octets (if password is longer than 16 octets process of securing is changed, password will be broken to chunks and encrypted in chain). So we take the dictionary (RFC 7511 as noted in hint) and try every word from it.
\\
Source code is in \textbf{break\_radius.py}. Captured traffic is in \textbf{traffic\_task2.pcap}. Found shared secret is \textbf{}.

\begin{thebibliography}{9}

\bibitem{radius}
  Remote Authentication Dial In User Service (RADIUS)
  \emph{\url{https://freeradius.org/rfc/rfc2865.html}}
    
\end{thebibliography}

\end{document} 
