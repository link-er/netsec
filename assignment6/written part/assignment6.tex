\documentclass{article}
\usepackage{graphicx}
\usepackage{hyperref}

\title{Assignment 6}
\author{Abbas Khan , Mariia Rybalka , Linara Adilova}
\begin{document}
\maketitle 

\section*{Task 6.1 (theoretical): Fast-Flux, Double-Flux}

\section*{Task 6.2 (theoretical): Buzzword Bingo}

\section*{Task 6.3 (theoretical): Firewalls}

\section *{Task 6.4 (theoretical) TOR}

\section *{Task 6.5 (practical): Simple Buffer Overflow}

\section*{Task 6.6 (bonus) Multiple Choice}
Q1: What ISO/OSI layers does a packet filter usually inspect? - 1) Layer 1 and 2
\\
Q2: What method can NOT be used during a TLS connection establishment (to an HTTPS webserver)? - 3) RADIUS
\\
Q3: Which key(s) belong into an X.509 certificate? - 2) The public key
\\
Q4: Consider you want to connect to a LAN that has 802.1X-controlled ports. What traffic is allowed to pass before a successful authentication? - 1) EAPoL

\section*{Task 6.7 (bonus) Citing Correctly}
"Centralized botnets are easy targets for takedown efforts by computer security researchers and law enforcement." \cite{botnets} However, there are also peer-to-peer botnets.
\\...\\
The authors propose a graph model to capture the vulnerabilities of P2P botnets and apply it several malware families in order to asses their resilience against different attacks \cite{botnets}. ...

\begin{thebibliography}{9}

\bibitem{botnets}
  SoK: P2PWNED - Modeling and Evaluating the Resilience of Peer-to-Peer Botnets
  \emph{Plohmann et al.}
  Fraunhofer FKIE, Bonn, Germany, daniel.plohmann@fkie.fraunhofer.de
    
\end{thebibliography}

\end{document} 
